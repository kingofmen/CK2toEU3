\documentclass[12pt,ebook,oneside]{book}

\usepackage{graphicx} 
\usepackage{layouts} 
\usepackage{hyperref} 
\textheight=1.2\textheight

\begin{document}
\chapter{Provinces}
The converter sets:
\begin{itemize}
\item Province ownership
\item Province control
\item Province basetax
\item Province manpower 
\item Province religion and culture.  
\end{itemize}

\section{Province ownership}

Ownership is decided by holding the County title. In the case
of several CK provinces mapping to one EU province, the decision
is by plurality of the CK titles weighted by base tax, with 
tiebreaking effectively random.

\section{Control}

Control is based on occupation of the baronies, including the
county seat. Unoccupied baronies count as held by the province 
owner. 

\section{Base tax}

Redistributed from the EU3 base tax, with weight equal to
the tax income of all holdings in each CK province corresponding
to the EU3 province. In cases where a CK province maps to more
than one EU3 province, the weight is divided equally between the
EU3 ones. Note that the tax income is not adjusted for technology. 

\section{Manpower}

Redistributed from EU3 manpower, with weight given by the levy
size of the CK holdings in each province; otherwise the algorithm
is the same as for base tax. Light infantry counts only
one-half in the weighting. 

\section{Religion and culture}

\label{sec:provculture}

Converted according to the mappings in \verb|religion_mappings.txt|
and \verb|culture_mappings.txt| respectively. In the case of multiple
CK provinces, weighted by base tax. In the case of culture, some large
CK2 cultures are broken up by region, such that, for example, `german'
converts differently depending on the de-jure duchy it is found in:
\begin{verbatim}
link = { ck2 = german eux = pommeranian de_jure = d_pommerania }
link = { ck2 = german eux = pommeranian de_jure = d_pomeralia }
link = { ck2 = german eux = pommeranian de_jure = d_mecklemburg }
link = { ck2 = german eux = prussian de_jure = d_livonia }
link = { ck2 = german eux = prussian de_jure = d_prussia }
link = { ck2 = german eux = prussian de_jure = d_polotsk }
\end{verbatim} 

\chapter{States}

The following state-level variables are handled:
\begin{itemize}
\item Vassals.
\item Government types. 
\item Prestige and gold. 
\item Culture. 
\item Religion. 
\end{itemize}

\section{Vassalisation}

The CK vassal structure is followed. If a CK primary title converts,
and it has a liege which also converts to a different tag, then the first EU nation
becomes the vassal of the second. 

\section{Government type}

Depends on the succession law. CK titles with law
\verb|patrician_elective| become merchant republics; all others
become feudal monarchies.

\section{Prestige and gold}

The EU3 gold is redistributed in simple proportion to
how much CK gold the converted ruler has, except that 
negative gold is counted as zero and the minimum amount
of gold is \verb|minimumGold|, to avoid instant loans. 

Prestige is converted in proportion to the sum of prestige and
piety. The constant of proportionality is such that the greatest
CK score (positive or negative) converts to the greatest EU prestige
(positive or negative), or the value of \verb|minimumMaxPrestige| in
the config file. 

\section{Culture}

State culture is determined by the CK culture of the ruler and all his
vassals, as follows. Each ruler is given some weight for each title he
has, depending on the tier:
\begin{itemize}
\item Empire, Kingdom - 1 point
\item Duchy, Barony - 1.5 points
\item County - 2 points
\end{itemize}
Note that counties are most highly weighted; this is to attempt to
reflect the makeup of the nobility of the country at large. The
emperor is a distant figure, as are Kings, while Barons are not very
powerful; it is the local Count who
determines whether the peasants are under a ``foreign yoke'' or not.
These weights can be set in \verb|config.txt|, where they are given 
as \verb|e_weight|, \verb|k_weight|, and so on. 

Once the title weights have been summed, they are divided by distance
from the sovereign ruler: Divide by 1 for the ruler himself, 2 for his
direct vassals, 3 for their vassals, and so on. 

Titles which have a ``special de-jure rule'' as described in section
\ref{sec:provculture} give weight to their special culture instead of
the most general one; the weight is the same. 

The culture with the most weight becomes the primary culture. 
Cultures with weight at least 20\% of the primary culture's weight
become accepted. 

\section{Religion}

Converted in the same way as state culture, except that there is no
special treatment of particular regions. 



\end{document}
