\documentclass[12pt,ebook,oneside]{book}

\usepackage{graphicx} 
\usepackage{layouts} 
\usepackage{hyperref} 
\textheight=1.2\textheight

\begin{document}
\chapter{Provinces}
The converter sets:
\begin{itemize}
\item Province ownership
\item Province control
\item Province manpower (not implemented)
\item Province basetax 
\end{itemize}

\section{Province ownership}

Ownership is decided by holding the County title. In the case
of several CK provinces mapping to one EU province, the decision
is by plurality of the CK titles, with tiebreaking effectively
random.

\section{Control}

Control is based on occupation of the baronies, including the
county seat. Unoccupied baronies count as held by the province 
owner. 

\section{Base tax}

Redistributed from the EU3 base tax, with weight equal to
the tax income of all holdings in each CK province corresponding
to the EU3 province. In cases where a CK province maps to more
than one EU3 province, the weight is divided equally between the
EU3 ones. Note that the tax income is not adjusted for technology. 

\chapter{States}

The following state-level variables are handled:
\begin{itemize}
\item Vassals.
\item Government types. 
\end{itemize}

\section{Vassalisation}

The CK vassal structure is followed. If a CK primary title converts,
and it has a liege which also converts to a different tag, then the first EU nation
becomes the vassal of the second. 

\section{Government type}

Depends on the succession law. CK titles with law
\verb|patrician_elective| become merchant republics; all others
become feudal monarchies. 

\end{document}
