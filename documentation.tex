\documentclass[12pt,ebook,oneside]{book}

\usepackage{graphicx} 
\usepackage{layouts} 
\usepackage{hyperref} 
\textheight=1.2\textheight

\begin{document}
\chapter{Provinces}
The converter sets:
\begin{itemize}
\item Province ownership
\item Province control
\item Province basetax
\item Province manpower 
\item Province religion and culture
\item Cores
\item Buildings
\item Discoveries
\end{itemize}

\section{Province ownership}

Ownership is decided by holding the County title. In the case
of several CK provinces mapping to one EU province, the decision
is by plurality of the CK titles weighted by base tax, with 
tiebreaking effectively random.

\section{Control}

Control is based on occupation of the baronies, including the
county seat. Unoccupied baronies count as held by the province 
owner. 

\section{Base tax}

Redistributed from the EU3 base tax, with weight equal to
the tax income of all holdings in each CK province corresponding
to the EU3 province. In cases where a CK province maps to more
than one EU3 province, the weight is divided equally between the
EU3 ones. Note that the tax income is not adjusted for technology. 

\section{Manpower}

Redistributed from EU3 manpower, with weight given by the levy
size of the CK holdings in each province; otherwise the algorithm
is the same as for base tax. Light infantry counts only
one-half in the weighting. 

\section{Religion and culture}

\label{sec:provculture}

Converted according to the mappings in \verb|religion_mappings.txt|
and \verb|culture_mappings.txt| respectively. In the case of multiple
CK provinces, weighted by base tax. In the case of culture, some large
CK2 cultures are broken up by region, such that, for example, `german'
converts differently depending on the de-jure duchy it is found in:
\begin{verbatim}
link = { ck2 = german eux = pommeranian de_jure = d_pommerania }
link = { ck2 = german eux = pommeranian de_jure = d_pomeralia }
link = { ck2 = german eux = pommeranian de_jure = d_mecklemburg }
link = { ck2 = german eux = prussian de_jure = d_livonia }
link = { ck2 = german eux = prussian de_jure = d_prussia }
link = { ck2 = german eux = prussian de_jure = d_polotsk }
\end{verbatim} 

\section{Other}

All buildings are removed. In addition, the history is cleaned of
events containing any of the keywords listed in
\verb|keywords_to_remove|
within \verb|config.txt| with the key \verb|province|. 

\section{Cores}

EU3 nations get cores on all CK provinces that are part of their
de-jure primary-tier titles plus their county-level demesne; that is, Emperors get cores on all their
Empire titles, Kings get cores on their Kingdom titles, and so
on. Baronies do not give cores. 

\section{Buildings}

All buildings except forts are removed. Forts are converted from the
CK fortifications: The total \verb|fort_level| of each CK province is
counted - note that all baronies contribute. EU provinces having a CK
province with more than 95\% of the maximum CK fort level then get a
level-3 fort; more than 85\%, a level-2 fort; and so on. These numbers
can be changed in the \verb|fortLevels| object of \verb|config.txt|. 

\section{Discoveries}

All provinces that have a CK equivalent are set to be discovered by
the western tech group. In addition, any tech groups listed as
\verb|removeFromDiscoveries| will be set to not have discovered the CK
area. 

\chapter{States}

The following state-level variables are handled:
\begin{itemize}
\item Vassals.
\item Government types. 
\item Culture. 
\item Religion. 
\item Monarchs. 
\item Sliders.
\item Armies and navies. 
\item Treaties. 
\item Miscellaneous minor variables. 
\end{itemize}

\section{Vassalisation}

The CK vassal structure is followed. If a CK primary title converts,
and it has a liege which also converts to a different tag, then the first EU nation
becomes the vassal of the second. 

\section{Government type}

Depends on the succession law. CK titles with law
\verb|patrician_elective| become merchant republics; all others
become feudal monarchies or empires, depending on the tier of their
primary title. 

\section{Culture}

State culture is determined by the CK culture of the ruler and all his
vassals, as follows. Each ruler is given some weight for each title he
has, depending on the tier:
\begin{itemize}
\item Empire, Kingdom - 1 point
\item Duchy, Barony - 1.5 points
\item County - 2 points
\end{itemize}
Note that counties are most highly weighted; this is to attempt to
reflect the makeup of the nobility of the country at large. The
emperor is a distant figure, as are Kings, while Barons are not very
powerful; it is the local Count who
determines whether the peasants are under a ``foreign yoke'' or not.
These weights can be set in \verb|config.txt|, where they are given 
as \verb|e_weight|, \verb|k_weight|, and so on. 

Once the title weights have been summed, they are divided by distance
from the sovereign ruler: Divide by 1 for the ruler himself, 2 for his
direct vassals, 3 for their vassals, and so on. 

Titles which have a ``special de-jure rule'' as described in section
\ref{sec:provculture} give weight to their special culture instead of
the most general one; the weight is the same. 

The culture with the most weight becomes the primary culture. 
Cultures with weight at least 20\% of the primary culture's weight
become accepted. 

\section{Religion}

Converted in the same way as state culture, except that there is no
special treatment of particular regions. 

\section{Monarchs}

Converted from the CK rulers. As a simplification, the ruler's birth
date is taken as the start date of his reign. The military score is
calculated as nine times the character's percentage rank (among
rulers) in trait-adjusted martial score. That is, the ruler with the
worst martial score gets 3 MIL\footnote{EU3 ADM scores run from 3 to 9.}; the ruler with the best score gets
9. Everyone else gets nine times their percentage of the distance
between the two. For example, if the best martial is 10 and the worst is
2, then a ruler with martial 4 is 25\% of the distance from worst to
best, and therefore gets a MIL score of 5.25, which is rounded down to
5. 

The admin score is calculated similarly, but with equal weights for
stewardship and learning; the diplomacy score gives equal weight to CK
diplomacy and intrigue. 

Heirs are generated from the oldest living son of the CK monarch, or
oldest living daughter if there are no sons; if there are no children,
no EU heir is created. The conversion is otherwise the same. 

\section{Sliders}

All sliders are set to be between the limits in \verb|config.txt|, 
eg \verb|pluto = { -5 0 }| indicates that converted aristocracy varies from -5
(the EU minimum) to 0 (halfway to full plutocracy). All effect sizes
are set in \verb|config.txt|; the below refers to the defaults. For
each slider, each CK ruler counts up all his effects, then divides by
the difference between highest and lowest, making a percentage. He
then gets a slider number which is that percentage of the way from the
lowest possible to the highest. For example, the median plutocrat 
gets Plutocracy=-2.5, which is rounded up to -2. All effects that
depend on titles (ie laws) are multiplied by the tier of the title,
with Barony being one. 

\begin{itemize}
\item Plutocracy: Each city barony in held counties gives a point, each castle
  subtracts one. High city levy laws add points, feudal levy laws subtract
  points; high feudal taxation adds, high city taxation
  subtracts. This is on the grounds that the people supplying armies are
  powerful while the people paying taxes are downtrodden. Finally,
  having the \verb|patrician_elective| succession law - indicating a
  merchant republic - gives a boost to plutocracy. 
\item Decentralisation: Each holding gives a point, meaning large
  empires are less centralised; further, the Crown Power laws increase
  centralisation. 
\item Narrowminded: Increased by church holdings, decreased by city
  holdings. Having Papal investiture also increases narrowmindedness. 
\item Free trade: Affected only by techs. The reasoning is that
  advanced technology allows governments to impose controls on trade,
  so that ``free trade'' is not an ideological but a default position
  taken by primitive countries. 
\item Defensive: Again technology, but in this case also buildings. Castle Infrastructure and Improved
  Keeps are defensive, while Tactics, Military Organisation and Siege
  Equipment are offensive. Knights are offensive, pikemen and fort
  levels are defensive. 
\item Naval: Galleys give points for naval. All holdings give points
  for land, cities less than the other two kinds. 
\item Quantity: Based on buildings. Knights and heavy infantry give
  points for quality, light infantry for quantity. 
\item Free subjects: Technology. Majesty and Legalism give serfdom,
  Popular Customs and Cultural Flexibility give freedom. 
\end{itemize}

\section{Armies and navies}

Armies are converted by giving one EU3 infantry regiment per
\verb|unitConversionRatio| retinue units, to a maximum of
\verb|maxArmySize|, placed in the capital. 

Navies are created in proportion to the number of galleys supplied by
the CK buildings in a nation. The total number of EU3 ships is equal
to the number in the input save; they are evenly distributed
between carracks and cogs. They are placed in EU3 provinces that:
\begin{itemize}
\item Convert from CK provinces supplying galleys
\item Are owned by the EU3 nation in question
\item Are not in the \verb|forbidShips| list - this allows for cases
  where CK coastal provinces convert to EU3 inland ones.
\end{itemize}
Where possible, EU3 provinces that contain navies in vanilla are
preferred. 

\section{Treaties}

All countries with the same dynasty are set to be allies unless they
are at war. If any child or sibling (counting only the father -
half-siblings through the mother do not count) of a CK ruler (including the ruler
himself) are married to children or siblings of another CK ruler, and
they are not at war, a royal marriage is created. 


\section{Minor variables}

The EU3 gold is redistributed in simple proportion to
how much CK gold the converted ruler has, except that 
negative gold is counted as zero and the minimum amount
of gold is \verb|minimumGold|, to avoid instant loans. 

Prestige is converted in proportion to the sum of prestige and
piety. The constant of proportionality is such that the greatest
CK score (positive or negative) converts to the greatest EU prestige
(positive or negative), or the value of \verb|minimumMaxPrestige| in
the config file. 

Capitals are set, where possible, to match the CK capital. If that
can't be done, the counties and baronies of the ruler are searched,
and the first one corresponding to an EU3 province owned by the nation
in question is used as a capital. If that doesn't work, all EU3
provinces are searched, the first one owned by the nation being used. 

Stability is set to zero, as is badboy. Missionaries, colonists, and
so on are set to zero. Manpower is set to 1000. All national ideas are
removed. 

Military unit types are set to western medievial infantry and cavalry,
carracks, galleys, and cogs. 

All three traditions are set to zero. 

Inflation is set to zero. 

Legitimacy is set to 100\%. 

The history is cleaned of
events containing any of the keywords listed in
\verb|keywords_to_remove|
within \verb|config.txt| with the key \verb|state|. 

Tech group is set to western. Production and army tech levels are found by calculating the
median base tax and manpower per holding in the CK provinces
of each EU3 nation; navy tech instead uses the average number of
galleys, since if the median were used many nations with long
coastlines but large interiors would get zero. The top quartile gets level-4 tech (where
manpower governs army tech, galleys naval, and base tax the three
others); the bottom quartile get level-2; and everyone else gets the
standard level-3 starting tech. The top and bottom levels are
adjustable in the \verb|techLevels| object of \verb|config.txt|. For
trade and government techs, the median level of, respectively, economy
and government CK techs are used instead. 

\chapter{Miscellany}

\section{The HRE}

The Holy Roman Empire is formed from the wealthiest empire in CK. If
there are no empires, no HRE is created. All vassals of the emperor
are made members of the HRE, and their land is made part of it; the 7
(by default; governed by \verb|numElectors| in \verb|config.txt|)
wealthiest are made electors. This will include the emperor himself
except in extremely unusual circumstances. 

If the field \verb|removeHRE| is set to 'yes', no HRE is created. 

\section{Centers of Trade}

COTs are created in the capital of each merchant republic. The owner
gets five merchants in it; ten merchants are distributed among those
other nations in which the CK owner (Doge only) has trade posts. This
number can be adjusted in the field \verb|numForeignMerchants| in
\verb|config.txt|. 

\section{Wars}

Wars are converted from CK wars if at least one character one each
side converted to an EU country as its sovereign. That is, if the Duke
of York is at war with Scotland, the war converts only if York is
sovereign in EU3; if York converts as part of England, England does
not start at war with Scotland. 

\section{Papacy}

The Catholic ruler with the highest piety is set as the Papal
Controller. Cardinals are created from the Catholic bishops with the
highest learning, and assigned to the owners of their respective
provinces. 

Papal influence is set proportional to piety, with the papal
controller (who is the most pious Catholic) getting the amount
\verb|maxPapalInfluence| from \verb|config.txt|. 

\section{Generals}

Each CK marshal becomes a general of his EU nation. (Note that where
several CK rulers were subsumed into one EU nation, there can be
multiple generals.) He gets the
square root of his Martial score as Shock, plus one to the listed
attribute for each trait he has that appears in the object
\verb|general_traits| in \verb|config.txt|. Traits that appear more
than once confer multiple stat increases. 

\section{Advisors}

Advisors are created from councilors who are not marshals. The type of
advisor is decided by, firstly, the council position; for example,
Chancellors can become the kind of advisors listed in the
\verb|job_chancellor| object in \verb|config.txt|. Second, among these
candidates the character gets points for attributes and traits, for
example:
\begin{verbatim}
grand_captain = {
  diplomacy = 1
  inspiring_leader = 10
} 
\end{verbatim}
indicates one point for each point of diplomacy, and 10 points for the
Inspiring Leader trait. The character becomes the advisor type with
the most points, of a level equal to the square root of the points. 
He is placed in the capital of his nation. 

Notice that not every EU3 advisor type is necessarily possible; the
default \verb|config.txt| leaves out navy-related, colonial-related,
and useless advisors. 
\end{document}
